
\chapter{rails迁移到subversion}

\section{在Linux下的迁移方法}
在命令行下输入:
\$ svnadmin create /path/to/repository

创建文件夹
\$ svn mkdir --message="Creating my project's repository ..."
file:///path/to/repository/trunk file:///path/to/repository/tags
file:///path/to/repository/branches

回到源代码所在目录
\$ cd /path/to/source/code

将代码导入subversion中
\$ svn import -m "initial import" . \
> file:///path/to/repository/trunk

现在可以进行如下操作
\$ mv code code.old
目的是把原有代码转移到其他目录,不要与原来的目录名字相同,在接下来的操
作中会发生冲突。如果对subversion完全放心的话,可以输入下面的命令:
\$ rm -rf code

把版本控制系统的代码导出来。
\$ svn checkout file:///path/to/repository/code/trunk code

现在进入刚导出的源代码文件夹
\$ cd code

忽略所有log和tmp文件夹下面的文件
\$ svn remove log/*
\$ svn commit -m 'removing all log files from subversion'
\$ svn propset svn:ignore "*.log" log/
\$ svn update log/
\$ svn commit -m 'Ignoring all files in /log/ ending in .log'
\$ svn remove tmp/*
\$ svn commit -m 'removing all tmp artifacts from subversion'
\$ svn propset svn:ignore "*" tmp/
\$ svn update tmp/
\$ svn commit -m "ignore tmp/ content from now on" 




